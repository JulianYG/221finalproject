\documentclass[a4paper]{article}

%% Language and font encodings
\usepackage[english]{babel}
\usepackage[utf8x]{inputenc}
\usepackage[T1]{fontenc}

%% Sets page size and margins
\usepackage[a4paper,top=3cm,bottom=2cm,left=3cm,right=3cm,marginparwidth=1.75cm]{geometry}

%% Useful packages
\usepackage{amsmath}
\usepackage{graphicx}
\usepackage[colorinlistoftodos]{todonotes}
\usepackage[colorlinks=true, allcolors=blue]{hyperref}

\title{ }
\author{Julian Gao (julianyg) Qiujiang Jin (qiujiang)}

\begin{document}
\maketitle

%\begin{abstract}
%Your abstract.
%\end{abstract}

\section{Introduction}

\section{Motivations}
In recent publications, CNN has shown strong ability in both geometry and semantic scene understanding. A lot of work has been done in indoor scene depth and normal estimation using single RGB image with pretty good results\cite{laina2016deeper}\cite{li2015depth}\cite{hane2015direction}\cite{eigen2015predicting}. But research on outdoor scene is hindered due to the difficulties on getting reliable ground truth for training. The depth ground truth of outdoor scene could not be acquired by Kinect, a device which is able to obtain depth maps with identical sampling rate to those of color images. Two widely-used depth measurements in outdoor environment are LiDAR and stereo. Stereo equipment is hard to get perfect calibration between multiple cameras and the process of computing depth from raw input is computationally expensive. Also the accuracy range of stereo is limited compared with LiDAR. On the other hand, although LiDAR has longer measuring distance, its sampling rate is extremely sparse. Considering the strong need of understanding geometry information in the outdoor road scene for autonomous driving, we decide to do single image depth estimation in outdoor road environment. 

\section{Potential Problems \& Approaches}
We plan to use on-the-shelf single indoor image depth estimation neural network structure from \cite{eigen2015predicting}. More specifically, we plan to use both stereo and LiDAR-captured depth information to generate ground truth for training. We expect to overcome the unsatisfactory depth ground truth problem, and the algorithm could predict depth from single outdoor road scene color image.
\\\\
Besides, since the geometry information is widely applicable, if the depth estimation task could be solved on schedule, it can be applied on car detection using geometry estimation: first reconstruct the 3D scene from depth prediction, then compute the probability map in the 3D scene showing where cars might be, and lastly use some on-the-shelf methods to detect cars in probability map.
 
\section{Dataset}
In this project, we choose to use the KITTI Benchmark \cite{Geiger2013IJRR}\cite{Fritsch2013ITSC}\cite{Menze2015CVPR}. It contains a large number of images with stereo and LiDAR sampled data, which can be used in the depth estimation task. For the detection problem, KITTI also has specific problem section for training and testing.

\bibliographystyle{ieeetr}
\bibliography{sample}

\end{document}